\documentclass[a4paper,10pt]{article}
\usepackage{eumat}

\begin{document}
\begin{eulernotebook}
\begin{eulercomment}
Nama : Ghifa Attaya Ulhaq\\
Kelas: Matematika B\\
NIM  : 22305144038

\begin{eulercomment}
\eulerheading{Menggambar Plot 3D dengan EMT}
\begin{eulercomment}
Ini adalah pengenalan plot 3D di Euler. Kita memerlukan plot 3D untuk
memvisualisasikan fungsi dua variabel.

Euler menggambar fungsi tersebut menggunakan algoritma pengurutan
untuk menyembunyikan bagian di latar belakang. Secara umum Euler
menggunakan proyeksi sentral. Defaultnya adalah dari kuadran x-y
positif menuju titik asal x=y=z=0, tetapi sudut=0° dilihat dari arah
sumbu y. Sudut pandang dan ketinggian dapat diubah.

Euler bisa merencanakan

- permukaan dengan garis penetasan dan level atau rentang level,\\
- awan titik,\\
- kurva parametrik,\\
- permukaan implisit.

Plot 3D suatu fungsi menggunakan plot3d. Cara termudah adalah dengan
memplot ekspresi dalam x dan y. Parameter r mengatur rentang plot
sekitar (0,0).
\end{eulercomment}
\begin{eulerprompt}
>aspect(1.5); plot3d("x^2+sin(y)",-5,5,0,6*pi):
\end{eulerprompt}
\eulerimg{17}{images/EMT4Plot3D_Ghifa Attaya Ulhaq_22305144038-001.png}
\begin{eulerprompt}
>plot3d("x^2+x*sin(y)",-5,5,0,6*pi):
\end{eulerprompt}
\eulerimg{17}{images/EMT4Plot3D_Ghifa Attaya Ulhaq_22305144038-002.png}
\begin{eulercomment}
Silakan lakukan modifikasi agar gambar "talang bergelombang" tersebut tidak lurus melainkan melengkung/melingkar, baik
melingkar secara mendatar maupun melingkar turun/naik (seperti papan peluncur pada kolam renang. Temukan rumusnya.
\end{eulercomment}
\eulerheading{Fungsi dari dua Variabel}
\begin{eulercomment}
Untuk grafik fungsi, gunakan

- ekspresi sederhana dalam x dan y,\\
- nama fungsi dari dua variabel\\
- atau matriks data.

Default dari fungsi ini adalah jaringan kawat yang diisi dengan warna
yang berbeda di kedua sisi. Perhatikan bahwa jumlah default interval
grid adalah 10, tapi plot menggunakan nomor default persegi panjang
40x40 untuk membangun permukaan. Tetapi, tetap dapat diubah.

- n = 40, n =[40,40]: jumlah garis grid di setiap arah\\
- grid = 10, grid =[10,10]: jumlah garis grid di setiap arah.

Kita menggunakan default n = 40 dan grid = 10.
\end{eulercomment}
\begin{eulerprompt}
>plot3d("x^2+y^2"):
\end{eulerprompt}
\eulerimg{17}{images/EMT4Plot3D_Ghifa Attaya Ulhaq_22305144038-003.png}
\begin{eulercomment}
Interaksi pengguna dimungkinkan dengan parameter pengguna. Pengguna
dapat menekan kunci berikut.

- kiri, kanan, atas, bawah: putar sudut pandang\\
- +,-: memperbesar atau keluar\\
- a: menghasilkan anaglyph (lihat di bawah)\\
- l: beralih memutar sumber cahaya (lihat di bawah)\\
- space: reset ke default\\
- return: interaksi akhir
\end{eulercomment}
\begin{eulerprompt}
>plot3d("exp(-x^2+y^2)",>user, ...
>  title="Turn with the vector keys (press return to finish)"):
\end{eulerprompt}
\eulerimg{17}{images/EMT4Plot3D_Ghifa Attaya Ulhaq_22305144038-004.png}
\begin{eulercomment}
Rentang plot untuk fungsi dapat ditentukan dengan

- a, b: jarak x\\
- c, d: Jarak Y\\
- r: alun-alun simetris sekitar (0,0).\\
- n: jumlah subinterval untuk plot.

Ada beberapa parameter untuk meningkatkan fungsi atau mengubah
tampilan grafik.

fscale: skala ke nilai fungsi (default adalah \textless{}fscale).\\
scale: nomor atau vektor 1x2 untuk skala menjadi x- dan arah-y.\\
frame: jenis bingkai (default 1).
\end{eulercomment}
\begin{eulerprompt}
>plot3d("exp(-(x^2+y^2)/5)",r=10,n=80,fscale=4,scale=1.2,frame=3,>user):
\end{eulerprompt}
\eulerimg{17}{images/EMT4Plot3D_Ghifa Attaya Ulhaq_22305144038-005.png}
\begin{eulercomment}
Pandangannya bisa diubah dengan berbagai cara.

- distance: jarak pandang ke plot.\\
- zoom: nilai zoom.\\
- angel: sudut ke sumbu-y negatif dalam radian.\\
- height: tinggi pemandangan dalam radian.

Nilai default dapat diperiksa atau diubah dengan fungsi view(). Ini
mengembalikan parameter dalam urutan di atas.
\end{eulercomment}
\begin{eulerprompt}
>view
\end{eulerprompt}
\begin{euleroutput}
  [5,  2.6,  2,  0.4]
\end{euleroutput}
\begin{eulercomment}
Jarak yang lebih dekat membutuhkan zoom yang lebih sedikit. Efeknya
lebih seperti lensa sudut lebar.

Dalam contoh berikut, angle= 0 dan height= 0 terlihat dari sumbu y
negatif. Label sumbu untuk y tersembunyi dalam kasus ini.
\end{eulercomment}
\begin{eulerprompt}
> plot3d("x^2+y",distance=3,zoom=1,angle=pi/2,height=0):
\end{eulerprompt}
\eulerimg{17}{images/EMT4Plot3D_Ghifa Attaya Ulhaq_22305144038-006.png}
\begin{eulercomment}
Plotnya selalu terlihat di tengah kubus plot. Anda dapat memindahkan
pusatnya dengan parameter tengah.
\end{eulercomment}
\begin{eulerprompt}
>plot3d("x^4+y^2",a=0,b=1,c=-1,d=1,angle=-20°,height=20°, ...
>  center=[0.4,0,0],zoom=5):
\end{eulerprompt}
\eulerimg{17}{images/EMT4Plot3D_Ghifa Attaya Ulhaq_22305144038-007.png}
\begin{eulercomment}
Plotnya ditingkatkan agar masuk ke dalam kubus unit untuk dilihat.
Jadi tidak perlu mengubah jarak atau zoom tergantung pada ukuran plot.
Labelnya mengacu pada ukuran sebenarnya.

Jika Anda mematikan ini dengan scale=false, Anda perlu berhati-hati,
bahwa plot masih cocok ke jendela plot, dengan mengubah jarak pandang
atau zoom, dan memindahkan pusat.
\end{eulercomment}
\begin{eulerprompt}
>plot3d("5*exp(-x^2-y^2)",r=2,<fscale,<scale,distance=13,height=50°, ...
>  center=[0,0,-2],frame=3):
\end{eulerprompt}
\eulerimg{17}{images/EMT4Plot3D_Ghifa Attaya Ulhaq_22305144038-008.png}
\begin{eulercomment}
Plot kutub juga tersedia. Parameter polar=true menarik plot kutub.
Fungsi tetap harus menjadi fungsi x dan y. Parameter "fscale" skala
fungsinya dengan skala sendiri. Jika tidak, fungsi ini ditingkatkan
agar masuk ke dalam kubus.
\end{eulercomment}
\begin{eulerprompt}
>plot3d("1/(x^2+y^2+1)",r=5,>polar, ...
>fscale=2,>hue,n=100,zoom=4,>contour,color=blue):
\end{eulerprompt}
\eulerimg{17}{images/EMT4Plot3D_Ghifa Attaya Ulhaq_22305144038-009.png}
\begin{eulerprompt}
>function f(r) := exp(-r/2)*cos(r); ...
>plot3d("f(x^2+y^2)",>polar,scale=[1,1,0.4],r=pi,frame=3,zoom=4):
\end{eulerprompt}
\eulerimg{17}{images/EMT4Plot3D_Ghifa Attaya Ulhaq_22305144038-010.png}
\begin{eulercomment}
Parameter berputar fungsi dalam x di sekitar sumbu x.

- rotate=1: Gunakan sumbu x\\
- rotate=2: Gunakan sumbu z
\end{eulercomment}
\begin{eulerprompt}
>plot3d("x^2+1",a=-1,b=1,rotate=true,grid=5):
\end{eulerprompt}
\eulerimg{17}{images/EMT4Plot3D_Ghifa Attaya Ulhaq_22305144038-011.png}
\begin{eulerprompt}
>plot3d("x^2+1",a=-1,b=1,rotate=2,grid=5):
\end{eulerprompt}
\eulerimg{17}{images/EMT4Plot3D_Ghifa Attaya Ulhaq_22305144038-012.png}
\begin{eulerprompt}
>plot3d("sqrt(25-x^2)",a=0,b=5,rotate=1):
\end{eulerprompt}
\eulerimg{17}{images/EMT4Plot3D_Ghifa Attaya Ulhaq_22305144038-013.png}
\begin{eulerprompt}
>plot3d("x*sin(x)",a=0,b=6pi,rotate=2):
\end{eulerprompt}
\eulerimg{17}{images/EMT4Plot3D_Ghifa Attaya Ulhaq_22305144038-014.png}
\begin{eulercomment}
Berikut adalah plot dengan tiga fungsi.
\end{eulercomment}
\begin{eulerprompt}
>plot3d("x","x^2+y^2","y",r=2,zoom=3.5,frame=3):
\end{eulerprompt}
\eulerimg{17}{images/EMT4Plot3D_Ghifa Attaya Ulhaq_22305144038-015.png}
\eulerheading{Plot kontur}
\begin{eulercomment}
Untuk plotnya, Euler menambahkan garis grid. Sebaliknya, mungkin untuk
menggunakan garis level dan rona satu warna atau rona berwarna
spektral. Euler dapat menggambar ketinggian fungsi pada plot dengan
bayangan. Dalam semua plot 3D Euler dapat menghasilkan anaglyph
merah/cyan.

- \textgreater{}hue: Mengaktifkan bayangan cahaya, bukan kabel.\\
- \textgreater{}contour: Plot garis kontur otomatis pada plot.\\
- level=... (atau levels): Sebuah vektor nilai untuk garis kontur.

Defaultnya adalah level="auto", yang menghitung beberapa garis level
secara otomatis. Seperti yang kau lihat di plot, tingkat sebenarnya
adalah tingkat tingkat.

Gaya default bisa diubah. Untuk plot kontur berikut, kita menggunakan
grid yang lebih halus untuk 100x100 poin, skala fungsi dan plot, dan
menggunakan sudut pandang yang berbeda.
\end{eulercomment}
\begin{eulerprompt}
>plot3d("exp(-x^2-y^2)",r=2,n=100,level="thin", ...
> >contour,>spectral,fscale=1,scale=1.1,angle=45°,height=20°):
\end{eulerprompt}
\eulerimg{17}{images/EMT4Plot3D_Ghifa Attaya Ulhaq_22305144038-016.png}
\begin{eulerprompt}
>plot3d("exp(x*y)",angle=100°,>contour,color=green):
\end{eulerprompt}
\eulerimg{17}{images/EMT4Plot3D_Ghifa Attaya Ulhaq_22305144038-017.png}
\begin{eulercomment}
Shading default menggunakan warna abu-abu. Tapi berbagai warna
spektral juga tersedia.

- \textgreater{}spectral: Digunakan skema spektral default\\
- color=...: Menggunakan warna khusus atau skema spektral

Untuk plot berikut, kita menggunakan skema spektral default dan
meningkatkan jumlah poin untuk mendapatkan tampilan yang sangat halus.
\end{eulercomment}
\begin{eulerprompt}
>plot3d("x^2+y^2",>spectral,>contour,n=100):
\end{eulerprompt}
\eulerimg{17}{images/EMT4Plot3D_Ghifa Attaya Ulhaq_22305144038-018.png}
\begin{eulercomment}
Alih-alih garis tingkat otomatis, kita juga dapat menetapkan nilai
garis level. Ini akan menghasilkan garis tingkat tipis bukannya
rentang level.
\end{eulercomment}
\begin{eulerprompt}
>plot3d("x^2-y^2",0,5,0,5,level=-1:0.1:1,color=redgreen):
\end{eulerprompt}
\eulerimg{17}{images/EMT4Plot3D_Ghifa Attaya Ulhaq_22305144038-019.png}
\begin{eulercomment}
Dalam plot berikut, kita menggunakan dua band tingkat yang sangat luas
dari -0.1 ke 1, dan dari 0.9 ke 1. Ini dimasukkan sebagai matriks
dengan batas tingkat sebagai kolom.

Selain itu, kita melapisi grid dengan 10 interval ke setiap arah.
\end{eulercomment}
\begin{eulerprompt}
>plot3d("x^2+y^3",level=[-0.1,0.9;0,1], ...
>  >spectral,angle=30°,grid=10,contourcolor=gray):
\end{eulerprompt}
\eulerimg{17}{images/EMT4Plot3D_Ghifa Attaya Ulhaq_22305144038-020.png}
\begin{eulercomment}
Dalam contoh berikut, kita merencanakan set, di mana

\end{eulercomment}
\begin{eulerformula}
\[
f(x,y) = x^y-y^x = 0
\]
\end{eulerformula}
\begin{eulercomment}
Kita menggunakan garis tipis tunggal untuk garis level.
\end{eulercomment}
\begin{eulerprompt}
>plot3d("x^y-y^x",level=0,a=0,b=6,c=0,d=6,contourcolor=red,n=100):
\end{eulerprompt}
\eulerimg{17}{images/EMT4Plot3D_Ghifa Attaya Ulhaq_22305144038-022.png}
\begin{eulercomment}
Hal ini dimungkinkan untuk menunjukkan pesawat kontur di bawah plot.
Warna dan jarak ke plot dapat ditentukan.
\end{eulercomment}
\begin{eulerprompt}
>plot3d("x^2+y^4",>cp,cpcolor=green,cpdelta=0.2):
\end{eulerprompt}
\eulerimg{17}{images/EMT4Plot3D_Ghifa Attaya Ulhaq_22305144038-023.png}
\begin{eulercomment}
Berikut adalah beberapa gaya lagi. Kita selalu mematikan bingkai, dan
menggunakan berbagai skema warna untuk plot dan grid.
\end{eulercomment}
\begin{eulerprompt}
>figure(2,2); ...
>expr="y^3-x^2"; ...
>figure(1);  ...
>  plot3d(expr,<frame,>cp,cpcolor=spectral); ...
>figure(2);  ...
>  plot3d(expr,<frame,>spectral,grid=10,cp=2); ...
>figure(3);  ...
>  plot3d(expr,<frame,>contour,color=gray,nc=5,cp=3,cpcolor=greenred); ...
>figure(4);  ...
>  plot3d(expr,<frame,>hue,grid=10,>transparent,>cp,cpcolor=gray); ...
>figure(0):
\end{eulerprompt}
\eulerimg{17}{images/EMT4Plot3D_Ghifa Attaya Ulhaq_22305144038-024.png}
\begin{eulercomment}
Ada beberapa skema spektral lainnya, bernomor dari 1 sampai 9. Tetapi
Anda juga bisa menggunakan color=value, dimana nilai

- spectral: untuk jangkauan dari biru ke merah\\
- white: untuk rentang yang lebih redup\\
- yellowblue,purplegreen,blueyellow,greenred\\
- blueyellow, greenpurple,yellowblue,redgreen
\end{eulercomment}
\begin{eulerprompt}
>figure(3,3); ...
>for i=1:9;  ...
>  figure(i); plot3d("x^2+y^2",spectral=i,>contour,>cp,<frame,zoom=4);  ...
>end; ...
>figure(0):
\end{eulerprompt}
\eulerimg{17}{images/EMT4Plot3D_Ghifa Attaya Ulhaq_22305144038-025.png}
\begin{eulercomment}
Sumber cahaya dapat diubah dengan l dan kunci kursor selama interaksi
pengguna. Hal ini juga dapat diatur dengan parameter.

- light: arah untuk cahaya\\
- amb: cahaya ambient antara 0 dan 1

Perhatikan bahwa program tidak membuat perbedaan antara sisi plot.
Tidak ada bayangan. Untuk ini Anda akan membutuhkan Povray.
\end{eulercomment}
\begin{eulerprompt}
>plot3d("-x^2-y^2", ...
>  hue=true,light=[0,1,1],amb=0,user=true, ...
>  title="Press l and cursor keys (return to exit)"):
\end{eulerprompt}
\eulerimg{17}{images/EMT4Plot3D_Ghifa Attaya Ulhaq_22305144038-026.png}
\begin{eulercomment}
Parameter warna mengubah warna permukaan. Warna garis level juga bisa
diubah.
\end{eulercomment}
\begin{eulerprompt}
>plot3d("-x^2-y^2",color=rgb(0.2,0.2,0),hue=true,frame=false, ...
>  zoom=3,contourcolor=red,level=-2:0.1:1,dl=0.01):
\end{eulerprompt}
\eulerimg{17}{images/EMT4Plot3D_Ghifa Attaya Ulhaq_22305144038-027.png}
\begin{eulercomment}
Warna 0 memberikan efek pelangi khusus.
\end{eulercomment}
\begin{eulerprompt}
>plot3d("x^2/(x^2+y^2+1)",color=0,hue=true,grid=10):
\end{eulerprompt}
\eulerimg{17}{images/EMT4Plot3D_Ghifa Attaya Ulhaq_22305144038-028.png}
\begin{eulercomment}
Permukaannya juga bisa transparan.
\end{eulercomment}
\begin{eulerprompt}
>plot3d("x^2+y^2",>transparent,grid=10,wirecolor=red):
\end{eulerprompt}
\eulerimg{17}{images/EMT4Plot3D_Ghifa Attaya Ulhaq_22305144038-029.png}
\eulerheading{Plot implisit}
\begin{eulercomment}
Pada tiga dimensi, juga terdapat plot implisit. Euler menghasilkan
potongan melalui benda. Fitur plot3d termasuk plot implisit. Plot ini
menunjukkan nol set fungsi dalam tiga variabel.\\
Solusi dari

\end{eulercomment}
\begin{eulerformula}
\[
f(x,y,z) = 0
\]
\end{eulerformula}
\begin{eulercomment}
dapat divisualisasikan dalam potongan paralel dengan x-y-, x-z- dan
bidang y-z.

- implicit=1: potong paralel dengan bidang y-z\\
- implicit=2: potong paralel dengan bidang x-z\\
- implicit=4: potong paralel dengan bidang x-y

Tambahkan nilai-nilai ini, jika Anda suka. Dalam contoh kita
merencanakan

\end{eulercomment}
\begin{eulerformula}
\[
M = \{ (x,y,z) : x^2+y^3+zy=1 \}
\]
\end{eulerformula}
\begin{eulerprompt}
>plot3d("x^2+y^3+z*y-1",r=5,implicit=3):
\end{eulerprompt}
\eulerimg{17}{images/EMT4Plot3D_Ghifa Attaya Ulhaq_22305144038-032.png}
\begin{eulerprompt}
>c=1; d=1;
>plot3d("((x^2+y^2-c^2)^2+(z^2-1)^2)*((y^2+z^2-c^2)^2+(x^2-1)^2)*((z^2+x^2-c^2)^2+(y^2-1)^2)-d",r=2,<frame,>implicit,>user): 
\end{eulerprompt}
\eulerimg{17}{images/EMT4Plot3D_Ghifa Attaya Ulhaq_22305144038-033.png}
\begin{eulerprompt}
>plot3d("x^2+y^2+4*x*z+z^3",>implicit,r=2,zoom=2.5):
\end{eulerprompt}
\eulerimg{17}{images/EMT4Plot3D_Ghifa Attaya Ulhaq_22305144038-034.png}
\eulerheading{Merencanakan Data 3D}
\begin{eulercomment}
Sama seperti plot2d, plot3d menerima data. Untuk objek 3D, Anda perlu
menyediakan matriks nilai dari x-, y- dan z-, atau tiga fungsi atau
ekspresi fx (x, y), fy (x, y), fz(x, y).

\end{eulercomment}
\begin{eulerformula}
\[
\gamma(t,s) = (x(t,s),y(t,s),z(t,s))
\]
\end{eulerformula}
\begin{eulercomment}
Karena x,y,z adalah matriks, kita berasumsi bahwa (t, s) berjalan
melalui grid persegi. Akibatnya, Anda dapat merencanakan gambar
persegi panjang di ruang angkasa.

Anda bisa menggunakan bahasa matriks Euler untuk menghasilkan
koordinat secara efektif.

Dalam contoh berikut, kita menggunakan vektor nilai t dan vektor kolom
nilai s untuk parameter permukaan bola. Dalam gambar, kita bisa
menandai daerah, dalam kasus kita wilayah kutub.
\end{eulercomment}
\begin{eulerprompt}
>t=linspace(0,2pi,180); s=linspace(-pi/2,pi/2,90)'; ...
>x=cos(s)*cos(t); y=cos(s)*sin(t); z=sin(s); ...
>plot3d(x,y,z,>hue, ...
>color=blue,<frame,grid=[10,20], ...
>values=s,contourcolor=red,level=[90°-24°;90°-22°], ...
>scale=1.4,height=50°):
\end{eulerprompt}
\eulerimg{17}{images/EMT4Plot3D_Ghifa Attaya Ulhaq_22305144038-036.png}
\begin{eulercomment}
Berikut adalah contoh, yang merupakan grafik fungsi.
\end{eulercomment}
\begin{eulerprompt}
>t=-1:0.1:1; s=(-1:0.1:1)'; plot3d(t,s,t*s,grid=10):
\end{eulerprompt}
\eulerimg{17}{images/EMT4Plot3D_Ghifa Attaya Ulhaq_22305144038-037.png}
\begin{eulercomment}
Namun, kita bisa membuat segala macam permukaan. Berikut adalah
permukaan yang sama dengan fungsi

\end{eulercomment}
\begin{eulerformula}
\[
x = y \, z
\]
\end{eulerformula}
\begin{eulerprompt}
>plot3d(t*s,t,s,angle=180°,grid=10):
\end{eulerprompt}
\eulerimg{17}{images/EMT4Plot3D_Ghifa Attaya Ulhaq_22305144038-039.png}
\begin{eulercomment}
Dengan lebih banyak usaha, kita bisa menghasilkan banyak permukaan.

Dalam contoh berikut kita membuat pandangan teduh dari bola
terdistorsi. Koordinat biasa untuk bola adalah

\end{eulercomment}
\begin{eulerformula}
\[
\gamma(t,s) = (\cos(t)\cos(s),\sin(t)\sin(s),\cos(s))
\]
\end{eulerformula}
\begin{eulercomment}
dengan

\end{eulercomment}
\begin{eulerformula}
\[
0 \le t \le 2\pi, \quad \frac{-\pi}{2} \le s \le \frac{\pi}{2}.
\]
\end{eulerformula}
\begin{eulercomment}
Kami mendistorsi ini dengan faktor

\end{eulercomment}
\begin{eulerformula}
\[
d(t,s) = \frac{\cos(4t)+\cos(8s)}{4}.
\]
\end{eulerformula}
\begin{eulerprompt}
>t=linspace(0,2pi,320); s=linspace(-pi/2,pi/2,160)'; ...
>d=1+0.2*(cos(4*t)+cos(8*s)); ...
>plot3d(cos(t)*cos(s)*d,sin(t)*cos(s)*d,sin(s)*d,hue=1, ...
>  light=[1,0,1],frame=0,zoom=5):
\end{eulerprompt}
\eulerimg{17}{images/EMT4Plot3D_Ghifa Attaya Ulhaq_22305144038-043.png}
\begin{eulercomment}
Tentu saja, point cloud juga mungkin. Untuk merencanakan data titik di
ruang, kita perlu tiga vektor untuk koordinat titik.

Gaya yang digunakan sama seperti di plot2d dengan points=true;
\end{eulercomment}
\begin{eulerprompt}
>n=500;  ...
>  plot3d(normal(1,n),normal(1,n),normal(1,n),points=true,style="."):
\end{eulerprompt}
\eulerimg{17}{images/EMT4Plot3D_Ghifa Attaya Ulhaq_22305144038-044.png}
\begin{eulercomment}
Hal ini juga mungkin untuk merencanakan kurva dalam 3D. Dalam hal ini,
lebih mudah untuk menghitung kembali titik kurva. Untuk kurva di
bidang kita menggunakan urutan koordinat dan parameter wire=true.
\end{eulercomment}
\begin{eulerprompt}
>t=linspace(0,8pi,500); ...
>plot3d(sin(t),cos(t),t/10,>wire,zoom=3):
\end{eulerprompt}
\eulerimg{17}{images/EMT4Plot3D_Ghifa Attaya Ulhaq_22305144038-045.png}
\begin{eulerprompt}
>t=linspace(0,4pi,1000); plot3d(cos(t),sin(t),t/2pi,>wire, ...
>linewidth=3,wirecolor=blue):
\end{eulerprompt}
\eulerimg{17}{images/EMT4Plot3D_Ghifa Attaya Ulhaq_22305144038-046.png}
\begin{eulerprompt}
>X=cumsum(normal(3,100)); ...
> plot3d(X[1],X[2],X[3],>anaglyph,>wire):
\end{eulerprompt}
\eulerimg{17}{images/EMT4Plot3D_Ghifa Attaya Ulhaq_22305144038-047.png}
\begin{eulercomment}
EMT juga bisa plot dalam mode anaglyph. Untuk melihat plot seperti
itu, Anda perlu kacamata merah/cyan.
\end{eulercomment}
\begin{eulerprompt}
> plot3d("x^2+y^3",>anaglyph,>contour,angle=30°):
\end{eulerprompt}
\eulerimg{17}{images/EMT4Plot3D_Ghifa Attaya Ulhaq_22305144038-048.png}
\begin{eulercomment}
Seringkali, skema warna spektral digunakan untuk plot. Ini menekankan
ketinggian fungsi.
\end{eulercomment}
\begin{eulerprompt}
>plot3d("x^2*y^3-y",>spectral,>contour,zoom=3.2):
\end{eulerprompt}
\eulerimg{17}{images/EMT4Plot3D_Ghifa Attaya Ulhaq_22305144038-049.png}
\begin{eulercomment}
Euler juga bisa merencanakan permukaan parameter, ketika parameternya
adalah nilai x-, y-, dan z dari gambar grid persegi panjang di ruang.

Untuk demo berikut, kita mengatur parameter u- dan v-, dan
menghasilkan koordinat ruang dari ini.
\end{eulercomment}
\begin{eulerprompt}
>u=linspace(-1,1,10); v=linspace(0,2*pi,50)'; ...
>X=(3+u*cos(v/2))*cos(v); Y=(3+u*cos(v/2))*sin(v); Z=u*sin(v/2); ...
>plot3d(X,Y,Z,>anaglyph,<frame,>wire,scale=2.3):
\end{eulerprompt}
\eulerimg{17}{images/EMT4Plot3D_Ghifa Attaya Ulhaq_22305144038-050.png}
\begin{eulercomment}
Berikut adalah contoh yang lebih rumit, yang megah dengan kacamata
merah/cyan.
\end{eulercomment}
\begin{eulerprompt}
>u:=linspace(-pi,pi,160); v:=linspace(-pi,pi,400)';  ...
>x:=(4*(1+.25*sin(3*v))+cos(u))*cos(2*v); ...
>y:=(4*(1+.25*sin(3*v))+cos(u))*sin(2*v); ...
> z=sin(u)+2*cos(3*v); ...
>plot3d(x,y,z,frame=0,scale=1.5,hue=1,light=[1,0,-1],zoom=2.8,>anaglyph):
\end{eulerprompt}
\eulerimg{17}{images/EMT4Plot3D_Ghifa Attaya Ulhaq_22305144038-051.png}
\eulerheading{Plot Statistik}
\begin{eulercomment}
Pada plot statistik, digunakan plot bar. Plot bar adalah jenis plot
statistik yang digunakan untuk menampilkan data kategori atau data
diskrit. Untuk ini, kita harus menyediakan

- x: vektor baris dengan elemen n + 1\\
- y: vektor kolom dengan elemen n + 1\\
- Z: matriks nxn dari nilai-nilai.

z bisa lebih besar, tapi hanya nilai nxn yang akan digunakan.

Dalam contoh, kita pertama menghitung nilai-nilai. Kemudian kita
menyesuaikan x dan y, sehingga pusat vektor pada nilai yang digunakan.
\end{eulercomment}
\begin{eulerprompt}
>x=-1:0.1:1; y=x'; z=x^2+y^2; ...
>xa=(x|1.1)-0.05; ya=(y_1.1)-0.05; ...
>plot3d(xa,ya,z,bar=true):
\end{eulerprompt}
\eulerimg{17}{images/EMT4Plot3D_Ghifa Attaya Ulhaq_22305144038-052.png}
\begin{eulercomment}
Hal ini dimungkinkan untuk membagi plot permukaan dalam dua atau lebih
bagian.
\end{eulercomment}
\begin{eulerprompt}
>x=-1:0.1:1; y=x'; z=x+y; d=zeros(size(x)); ...
>plot3d(x,y,z,disconnect=2:2:20):
\end{eulerprompt}
\eulerimg{17}{images/EMT4Plot3D_Ghifa Attaya Ulhaq_22305144038-053.png}
\begin{eulercomment}
Jika memuat atau menghasilkan matriks data M dari file dan perlu plot
dalam 3D Anda dapat baik skala matriks ke [-1,1] dengan scale(M), atau
skala matriks dengan \textgreater{}zscale. Ini dapat dikombinasikan dengan faktor
penskalaan individu yang diterapkan juga.
\end{eulercomment}
\begin{eulerprompt}
>i=1:20; j=i'; ...
>plot3d(i*j^2+100*normal(20,20),>zscale,scale=[1,1,1.5],angle=-40°,zoom=1.8):
\end{eulerprompt}
\eulerimg{17}{images/EMT4Plot3D_Ghifa Attaya Ulhaq_22305144038-054.png}
\begin{eulerprompt}
>Z=intrandom(5,100,6); v=zeros(5,6); ...
>loop 1 to 5; v[#]=getmultiplicities(1:6,Z[#]); end; ...
>columnsplot3d(v',scols=1:5,ccols=[1:5]):
\end{eulerprompt}
\eulerimg{17}{images/EMT4Plot3D_Ghifa Attaya Ulhaq_22305144038-055.png}
\eulerheading{Permukaan Benda Putar}
\begin{eulerprompt}
>plot2d("(x^2+y^2-1)^3-x^2*y^3",r=1.3, ...
>style="#",color=red,<outline, ...
>level=[-2;0],n=100):
\end{eulerprompt}
\eulerimg{17}{images/EMT4Plot3D_Ghifa Attaya Ulhaq_22305144038-056.png}
\begin{eulerprompt}
>ekspresi &= (x^2+y^2-1)^3-x^2*y^3; $ekspresi
\end{eulerprompt}
\begin{eulerformula}
\[
\left(y^2+x^2-1\right)^3-x^2\,y^3
\]
\end{eulerformula}
\begin{eulercomment}
Kita ingin mengubah kurva jantung di sekitar sumbu y. Berikut adalah
ekspresi, yang mendefinisikan jantung:

\end{eulercomment}
\begin{eulerformula}
\[
f(x,y)=(x^2+y^2-1)^3-x^2.y^3.
\]
\end{eulerformula}
\begin{eulercomment}
Selanjutnya kita mengatur

\end{eulercomment}
\begin{eulerformula}
\[
x=r.cos(a),\quad y=r.sin(a).
\]
\end{eulerformula}
\begin{eulerprompt}
>function fr(r,a) &= ekspresi with [x=r*cos(a),y=r*sin(a)] | trigreduce; $fr(r,a)
\end{eulerprompt}
\begin{eulerformula}
\[
\left(r^2-1\right)^3+\frac{\left(\sin \left(5\,a\right)-\sin \left(  3\,a\right)-2\,\sin a\right)\,r^5}{16}
\]
\end{eulerformula}
\begin{eulercomment}
Hal ini memungkinkan untuk menentukan fungsi numerik, yang memecahkan
untuk r, jika diberikan. Dengan fungsi itu kita bisa merencanakan
jantung berubah sebagai permukaan parametrik.
\end{eulercomment}
\begin{eulerprompt}
>function map f(a) := bisect("fr",0,2;a); ...
>t=linspace(-pi/2,pi/2,100); r=f(t);  ...
>s=linspace(pi,2pi,100)'; ...
>plot3d(r*cos(t)*sin(s),r*cos(t)*cos(s),r*sin(t), ...
>>hue,<frame,color=red,zoom=4,amb=0,max=0.7,grid=12,height=50°):
\end{eulerprompt}
\eulerimg{17}{images/EMT4Plot3D_Ghifa Attaya Ulhaq_22305144038-061.png}
\begin{eulercomment}
Berikut ini adalah plot 3D dari gambar di atas diputar di sekitar
sumbu z. Kita mendefinisikan fungsi, yang menggambarkan objek.
\end{eulercomment}
\begin{eulerprompt}
>function f(x,y,z) ...
\end{eulerprompt}
\begin{eulerudf}
  r=x^2+y^2;
  return (r+z^2-1)^3-r*z^3;
   endfunction
\end{eulerudf}
\begin{eulerprompt}
>plot3d("f(x,y,z)", ...
>xmin=0,xmax=1.2,ymin=-1.2,ymax=1.2,zmin=-1.2,zmax=1.4, ...
>implicit=1,angle=-30°,zoom=2.5,n=[10,100,60],>anaglyph):
\end{eulerprompt}
\eulerimg{17}{images/EMT4Plot3D_Ghifa Attaya Ulhaq_22305144038-062.png}
\eulerheading{Plot 3D khusus}
\begin{eulercomment}
Fungsi plot3d bagus untuk dimiliki, tapi tidak memenuhi semua
kebutuhan. Selain rutinitas dasar, adalah mungkin untuk mendapatkan
plot berbingkai dari objek yang Anda suka.

Meskipun Euler bukan program 3D, itu dapat menggabungkan beberapa
objek dasar. Kita akan mencoba memvisualisasikan paraboloid dan
singgung nya.
\end{eulercomment}
\begin{eulerprompt}
>function myplot ...
\end{eulerprompt}
\begin{eulerudf}
    y=-1:0.01:1; x=(-1:0.01:1)';
    plot3d(x,y,0.2*(x-0.1)/2,<scale,<frame,>hue, ..
      hues=0.5,>contour,color=orange);
    h=holding(1);
    plot3d(x,y,(x^2+y^2)/2,<scale,<frame,>contour,>hue);
    holding(h);
  endfunction
\end{eulerudf}
\begin{eulercomment}
Sekarang framedplot() menyediakan bingkai, dan mengatur pemandangan.
\end{eulercomment}
\begin{eulerprompt}
>framedplot("myplot",[-1,1,-1,1,0,1],height=0,angle=-30°, ...
>  center=[0,0,-0.7],zoom=3):
\end{eulerprompt}
\eulerimg{17}{images/EMT4Plot3D_Ghifa Attaya Ulhaq_22305144038-063.png}
\begin{eulercomment}
Dengan cara yang sama, Anda dapat merencanakan bidang kontur secara
manual. Perhatikan bahwa plot3d() mengatur jendela ke fullwindow()
secara default, tapi plotcontourplane() mengasumsikan itu.
\end{eulercomment}
\begin{eulerprompt}
>x=-1:0.02:1.1; y=x'; z=x^2-y^4;
>function myplot (x,y,z) ...
\end{eulerprompt}
\begin{eulerudf}
    zoom(2);
    wi=fullwindow();
    plotcontourplane(x,y,z,level="auto",<scale);
    plot3d(x,y,z,>hue,<scale,>add,color=white,level="thin");
    window(wi);
    reset();
  endfunction
\end{eulerudf}
\begin{eulerprompt}
>myplot(x,y,z):
\end{eulerprompt}
\eulerimg{27}{images/EMT4Plot3D_Ghifa Attaya Ulhaq_22305144038-064.png}
\eulerheading{Animasi}
\begin{eulercomment}
Euler dapat menggunakan bingkai untuk menghitung animasi.

Salah satu fungsi, yang menggunakan teknik ini adalah berputar. Ini
dapat mengubah sudut pandang dan menggambar ulang plot 3D. Fungsi ini
memanggil addpage() untuk setiap plot baru. Akhirnya itu menghidupkan
plotnya.

Silakan mempelajari sumber putaran (the source of rotate) untuk
melihat lebih detail.
\end{eulercomment}
\begin{eulerprompt}
>function testplot () := plot3d("x^2+y^3"); ...
>rotate("testplot"); testplot():
\end{eulerprompt}
\eulerimg{27}{images/EMT4Plot3D_Ghifa Attaya Ulhaq_22305144038-065.png}
\eulerheading{Menggambar Povray}
\begin{eulercomment}
Dengan bantuan file Euler povray.e, Euler dapat menghasilkan file
Povray. Hasilnya sangat bagus untuk dilihat.

Anda perlu menginstal Povray (32bit atau 64bit) dari
http://www.povray.org/ , dan menempatkan sub-direktori "bin" Povray ke jalur lingkungan, atau mengatur variabel "defaultpovray" dengan jalur penuh menunjuk ke "pvengine.exe".

Antarmuka Povray dari Euler menghasilkan file Povray di direktori
rumah pengguna, dan memanggil Povray untuk mengurai file-file ini.
Nama file default adalah current.pov, dan direktori default adalah
eulerhome(), biasanya c:\textbackslash{}User\textbackslash{}User\textbackslash{}Euler. Povray menghasilkan file
PNG, yang dapat dimuat oleh Euler ke dalam buku catatan. Untuk
membersihkan file-file ini, gunakan povclear().

Fungsi pov3d memiliki dukungan yang sama dengan plot3d. Ini dapat
menghasilkan grafik fungsi f(x,y), atau permukaan dengan koordinat X,
Y, Z dalam matriks, termasuk garis tingkat opsional. Fungsi ini
memulai raytracer secara otomatis, dan memuat adegan ke dalam buku
catatan Euler.

Selain pov3d(), ada banyak fungsi, yang menghasilkan objek Povray.
Fungsi ini mengembalikan string, berisi kode Povray untuk objek. Untuk
menggunakan fungsi ini, mulai file Povray dengan povstart(). Kemudian
gunakan writeln(...) untuk menulis objek ke file adegan. Akhirnya,
akhiri file dengan povend(). Secara default, raytracer akan mulai, dan
PNG akan dimasukkan ke dalam notebook Euler.

Fungsi objek memiliki parameter yang disebut, yang membutuhkan string
dengan kode Povray untuk tekstur dan akhir objek. Fungsi povlook()
dapat digunakan untuk menghasilkan string ini. Ini memiliki parameter
untuk warna, transparansi, Phong Shading dll.

Perhatikan bahwa seluruh bidang Povray memiliki sistem koordinat lain.
Antarmuka ini menerjemahkan semua koordinat ke sistem Povray. Jadi
Anda dapat terus berpikir dalam sistem koordinat Euler dengan z
menunjuk vertikal ke atas, dan sumbu x,y,z dalam arti tangan kanan.\\
Anda harus memuat file povray.
\end{eulercomment}
\begin{eulerprompt}
>load povray;
\end{eulerprompt}
\begin{eulercomment}
Pastikan, direktori Povray bin ada di jalurnya. Jika tidak, edit
variabel berikut sehingga berisi jalur ke povray yang dapat
dieksekusi.
\end{eulercomment}
\begin{eulerprompt}
>defaultpovray="C:\(\backslash\)Program Files\(\backslash\)POV-Ray\(\backslash\)v3.7\(\backslash\)bin\(\backslash\)pvengine.exe"
\end{eulerprompt}
\begin{euleroutput}
  C:\(\backslash\)Program Files\(\backslash\)POV-Ray\(\backslash\)v3.7\(\backslash\)bin\(\backslash\)pvengine.exe
\end{euleroutput}
\begin{eulercomment}
Untuk kesan pertama, kita membuat plot dari sebuah fungsi sederhana.
Perintah berikut menghasilkan sebuah file povray di direktori pengguna
Anda, dan menjalankan Povray untuk melakukan ray tracing terhadap file
ini.

Jika Anda menjalankan perintah berikut, GUI Povray seharusnya akan
terbuka, menjalankan file, dan secara otomatis menutupnya. Karena
alasan keamanan, Anda akan ditanya apakah Anda ingin mengizinkan file
exe ini untuk berjalan. Anda dapat menekan batal untuk menghentikan
pertanyaan lebih lanjut. Anda mungkin perlu menekan OK di jendela
Povray untuk mengakui dialog awal Povray.
\end{eulercomment}
\begin{eulerprompt}
>plot3d("x^2+y^2",zoom=2):
\end{eulerprompt}
\eulerimg{27}{images/EMT4Plot3D_Ghifa Attaya Ulhaq_22305144038-066.png}
\begin{eulerprompt}
>pov3d("x^2+y^2",zoom=3);
\end{eulerprompt}
\eulerimg{27}{images/EMT4Plot3D_Ghifa Attaya Ulhaq_22305144038-067.png}
\begin{eulercomment}
Kita dapat membuat fungsinya transparan dan menambahkan penyelesaian
lainnya. Kita juga dapat menambahkan garis level ke plot fungsi.
\end{eulercomment}
\begin{eulerprompt}
>pov3d("x^2+y^3",axiscolor=red,angle=-45°,>anaglyph, ...
>  look=povlook(cyan,0.2),level=-1:0.5:1,zoom=3.8);
\end{eulerprompt}
\eulerimg{27}{images/EMT4Plot3D_Ghifa Attaya Ulhaq_22305144038-068.png}
\begin{eulercomment}
Terkadang perlu untuk mencegah penskalaan fungsi, dan menskalakan
fungsi secara manual.

Kita memplot himpunan titik pada bidang kompleks, dimana hasil kali
jarak ke 1 dan -1 sama dengan 1.
\end{eulercomment}
\begin{eulerprompt}
>pov3d("((x-1)^2+y^2)*((x+1)^2+y^2)/40",r=2, ...
>  angle=-120°,level=1/40,dlevel=0.005,light=[-1,1,1],height=10°,n=50, ...
>  <fscale,zoom=3.8);
\end{eulerprompt}
\eulerimg{27}{images/EMT4Plot3D_Ghifa Attaya Ulhaq_22305144038-069.png}
\eulerheading{Plotting dengan Koordinat}
\begin{eulercomment}
Daripada menggunakan fungsi, kita dapat melakukan plotting dengan
menggunakan koordinat. Seperti pada plot3d, kita memerlukan tiga
matriks untuk mendefinisikan objek.

Pada contoh ini, kita memutar sebuah fungsi sekitar sumbu z.
\end{eulercomment}
\begin{eulerprompt}
>function f(x) := x^3-x+1; ...
>x=-1:0.01:1; t=linspace(0,2pi,50)'; ...
>Z=x; X=cos(t)*f(x); Y=sin(t)*f(x); ...
>pov3d(X,Y,Z,angle=40°,look=povlook(red,0.1),height=50°,axis=0,zoom=4,light=[10,5,15]);
\end{eulerprompt}
\eulerimg{27}{images/EMT4Plot3D_Ghifa Attaya Ulhaq_22305144038-070.png}
\begin{eulercomment}
Dalam contoh berikut, kita memplot gelombang teredam. Kita
menghasilkan gelombang tersebut dengan bahasa matriks Euler.

Kita juga menunjukkan bagaimana objek tambahan dapat ditambahkan ke
dalam sebuah adegan pov3d. Untuk menghasilkan objek-objek tersebut,
lihat contoh-contoh berikutnya. Perlu diperhatikan bahwa plot3d
mengubah skala plot sehingga cocok dalam kubus satuan.
\end{eulercomment}
\begin{eulerprompt}
>r=linspace(0,1,80); phi=linspace(0,2pi,80)'; ...
>x=r*cos(phi); y=r*sin(phi); z=exp(-5*r)*cos(8*pi*r)/3;  ...
>pov3d(x,y,z,zoom=6,axis=0,height=30°,add=povsphere([0.5,0,0.25],0.15,povlook(red)), ...
>  w=500,h=300);
\end{eulerprompt}
\eulerimg{16}{images/EMT4Plot3D_Ghifa Attaya Ulhaq_22305144038-071.png}
\begin{eulercomment}
Dengan metode shading canggih Povray, hanya sedikit titik dapat
menghasilkan permukaan yang sangat halus. Hanya di batas-batas dan
dalam bayangan trik ini mungkin menjadi jelas.

Untuk ini, kita perlu menambahkan vektor normal di setiap titik
matriks.
\end{eulercomment}
\begin{eulerprompt}
>Z &= x^2*y^3
\end{eulerprompt}
\begin{euleroutput}
  
                                   2  3
                                  x  y
  
\end{euleroutput}
\begin{eulercomment}
Persamaan permukaannya adalah [x,y,Z]. Kami menghitung dua turunan
dari x dan y dan mengambil perkalian silangnya sebagai normal.
\end{eulercomment}
\begin{eulerprompt}
>dx &= diff([x,y,Z],x); dy &= diff([x,y,Z],y);
\end{eulerprompt}
\begin{eulercomment}
Kami mendefinisikan normal sebagai produk silang dari turunan ini, dan
mendefinisikan fungsi koordinat.
\end{eulercomment}
\begin{eulerprompt}
>N &= crossproduct(dx,dy); NX &= N[1]; NY &= N[2]; NZ &= N[3]; N,
\end{eulerprompt}
\begin{euleroutput}
  
                                 3       2  2
                         [- 2 x y , - 3 x  y , 1]
  
\end{euleroutput}
\begin{eulercomment}
Kami hanya menggunakan 25 poin.
\end{eulercomment}
\begin{eulerprompt}
>x=-1:0.5:1; y=x';
>pov3d(x,y,Z(x,y),angle=10°, ...
>  xv=NX(x,y),yv=NY(x,y),zv=NZ(x,y),<shadow);
\end{eulerprompt}
\eulerimg{27}{images/EMT4Plot3D_Ghifa Attaya Ulhaq_22305144038-072.png}
\begin{eulercomment}
Berikut adalah simpul Trefoil yang dibuat oleh A. Busser dalam Povray.
Terdapat versi yang ditingkatkan dari ini dalam contoh-contoh.

See: Examples\textbackslash{}Trefoil Knot \textbar{} Trefoil Knot

Untuk tampilan yang baik dengan tidak terlalu banyak titik, kami
menambahkan vektor normal di sini. Kami menggunakan Maxima untuk
menghitung vektor normal untuk kami. Pertama, tiga fungsi koordinat
sebagai ekspresi simbolis.
\end{eulercomment}
\begin{eulerprompt}
>X &= ((4+sin(3*y))+cos(x))*cos(2*y); ...
>Y &= ((4+sin(3*y))+cos(x))*sin(2*y); ...
>Z &= sin(x)+2*cos(3*y);
\end{eulerprompt}
\begin{eulercomment}
Kemudian kedua vektor turunan ke x dan y.
\end{eulercomment}
\begin{eulerprompt}
>dx &= diff([X,Y,Z],x); dy &= diff([X,Y,Z],y);
\end{eulerprompt}
\begin{eulercomment}
Sekarang normalnya, yaitu perkalian silang kedua turunannya.
\end{eulercomment}
\begin{eulerprompt}
>dn &= crossproduct(dx,dy);
\end{eulerprompt}
\begin{eulercomment}
Kita sekarang mengevaluasi semua ini secara numerik.
\end{eulercomment}
\begin{eulerprompt}
>x:=linspace(-%pi,%pi,40); y:=linspace(-%pi,%pi,100)';
\end{eulerprompt}
\begin{eulercomment}
Vektor normal adalah evaluasi ekspresi simbolik dn[i] untuk i=1,2,3.
Sintaksnya adalah \&"expression"(parameter). Ini adalah alternatif dari
metode pada contoh sebelumnya, di mana kita mendefinisikan ekspresi
simbolik NX, NY, NZ terlebih dahulu.
\end{eulercomment}
\begin{eulerprompt}
>pov3d(X(x,y),Y(x,y),Z(x,y),>anaglyph,axis=0,zoom=5,w=450,h=350, ...
>  <shadow,look=povlook(blue), ...
>  xv=&"dn[1]"(x,y), yv=&"dn[2]"(x,y), zv=&"dn[3]"(x,y));
\end{eulerprompt}
\eulerimg{21}{images/EMT4Plot3D_Ghifa Attaya Ulhaq_22305144038-073.png}
\begin{eulercomment}
Kita juga dapat menghasilkan grid dalam 3D.
\end{eulercomment}
\begin{eulerprompt}
>povstart(zoom=4); ...
>x=-1:0.5:1; r=1-(x+1)^2/6; ...
>t=(0°:30°:360°)'; y=r*cos(t); z=r*sin(t); ...
>writeln(povgrid(x,y,z,d=0.02,dballs=0.05)); ...
>povend();
\end{eulerprompt}
\eulerimg{27}{images/EMT4Plot3D_Ghifa Attaya Ulhaq_22305144038-074.png}
\begin{eulercomment}
Dengan povgrid(), kurva dimungkinkan.
\end{eulercomment}
\begin{eulerprompt}
>povstart(center=[0,0,1],zoom=3.6); ...
>t=linspace(0,2,1000); r=exp(-t); ...
>x=cos(2*pi*10*t)*r; y=sin(2*pi*10*t)*r; z=t; ...
>writeln(povgrid(x,y,z,povlook(red))); ...
>writeAxis(0,2,axis=3); ...
>povend();
\end{eulerprompt}
\eulerimg{27}{images/EMT4Plot3D_Ghifa Attaya Ulhaq_22305144038-075.png}
\eulerheading{Objek Povray}
\begin{eulercomment}
Di atas, kita menggunakan pov3d untuk memplot permukaan. Antarmuka
povray di Euler juga dapat menghasilkan objek Povray. Objek ini
disimpan sebagai string di Euler, dan perlu ditulis ke file Povray.

Kita memulai output dengan povstart().
\end{eulercomment}
\begin{eulerprompt}
>povstart(zoom=4);
\end{eulerprompt}
\begin{eulercomment}
Pertama kita mendefinisikan tiga silinder, dan menyimpannya dalam
string di Euler.

Fungsi povx() dll. hanya mengembalikan vektor [1,0,0], yang dapat
digunakan sebagai gantinya.
\end{eulercomment}
\begin{eulerprompt}
>c1=povcylinder(-povx,povx,1,povlook(red)); ...
>c2=povcylinder(-povy,povy,1,povlook(yellow)); ...
>c3=povcylinder(-povz,povz,1,povlook(blue)); ...
\end{eulerprompt}
\begin{eulercomment}
String tersebut berisi kode Povray, yang tidak perlu kita pahami pada
saat itu.
\end{eulercomment}
\begin{eulerprompt}
>c2
\end{eulerprompt}
\begin{euleroutput}
  cylinder \{ <0,0,-1>, <0,0,1>, 1
   texture \{ pigment \{ color rgb <0.941176,0.941176,0.392157> \}  \} 
   finish \{ ambient 0.2 \} 
   \}
\end{euleroutput}
\begin{eulercomment}
Seperti yang Anda lihat, kita menambahkan tekstur pada objek dalam
tiga warna berbeda.

Hal ini dilakukan oleh povlook(), yang mengembalikan string dengan
kode Povray yang relevan. Kita dapat menggunakan warna default Euler,
atau menentukan warna kita sendiri. Kita juga dapat menambahkan
transparansi, atau mengubah cahaya sekitar.
\end{eulercomment}
\begin{eulerprompt}
>povlook(rgb(0.1,0.2,0.3),0.1,0.5)
\end{eulerprompt}
\begin{euleroutput}
   texture \{ pigment \{ color rgbf <0.101961,0.2,0.301961,0.1> \}  \} 
   finish \{ ambient 0.5 \} 
  
\end{euleroutput}
\begin{eulercomment}
Sekarang kita mendefinisikan objek persimpangan, dan menulis hasilnya
ke file.
\end{eulercomment}
\begin{eulerprompt}
>writeln(povintersection([c1,c2,c3]));
\end{eulerprompt}
\begin{eulercomment}
Persimpangan tiga silinder sulit untuk divisualisasikan jika Anda
belum pernah melihatnya sebelumnya.
\end{eulercomment}
\begin{eulerprompt}
>povend;
\end{eulerprompt}
\eulerimg{27}{images/EMT4Plot3D_Ghifa Attaya Ulhaq_22305144038-076.png}
\begin{eulercomment}
Fungsi berikut menghasilkan fraktal secara rekursif.

Fungsi pertama menunjukkan bagaimana Euler menangani objek Povray
sederhana. Fungsi povbox() mengembalikan string, yang berisi koordinat
kotak, tekstur, dan hasil akhir.
\end{eulercomment}
\begin{eulerprompt}
>function onebox(x,y,z,d) := povbox([x,y,z],[x+d,y+d,z+d],povlook());
>function fractal (x,y,z,h,n) ...
\end{eulerprompt}
\begin{eulerudf}
   if n==1 then writeln(onebox(x,y,z,h));
   else
     h=h/3;
     fractal(x,y,z,h,n-1);
     fractal(x+2*h,y,z,h,n-1);
     fractal(x,y+2*h,z,h,n-1);
     fractal(x,y,z+2*h,h,n-1);
     fractal(x+2*h,y+2*h,z,h,n-1);
     fractal(x+2*h,y,z+2*h,h,n-1);
     fractal(x,y+2*h,z+2*h,h,n-1);
     fractal(x+2*h,y+2*h,z+2*h,h,n-1);
     fractal(x+h,y+h,z+h,h,n-1);
   endif;
  endfunction
\end{eulerudf}
\begin{eulerprompt}
>povstart(fade=10,<shadow);
>fractal(-1,-1,-1,2,4);
>povend();
\end{eulerprompt}
\eulerimg{27}{images/EMT4Plot3D_Ghifa Attaya Ulhaq_22305144038-077.png}
\begin{eulercomment}
Perbedaan memungkinkan pemisahan satu objek dari objek lainnya.
Seperti persimpangan, ada bagian dari objek CSG di Povray.
\end{eulercomment}
\begin{eulerprompt}
>povstart(light=[5,-5,5],fade=10);
\end{eulerprompt}
\begin{eulercomment}
Untuk demonstrasi ini, kita mendefinisikan sebuah objek dalam Povray,
daripada menggunakan sebuah string dalam Euler. Definisi ditulis ke
file secara langsung.

Koordinat sebuah kotak -1 hanya berarti [-1,-1,-1].
\end{eulercomment}
\begin{eulerprompt}
>povdefine("mycube",povbox(-1,1));
\end{eulerprompt}
\begin{eulercomment}
Kita bisa menggunakan objek ini di povobject(), yang mengembalikan
string seperti biasa.
\end{eulercomment}
\begin{eulerprompt}
>c1=povobject("mycube",povlook(red));
\end{eulerprompt}
\begin{eulercomment}
Kami membuat kubus kedua, dan memutar serta menskalakannya sedikit.
\end{eulercomment}
\begin{eulerprompt}
>c2=povobject("mycube",povlook(yellow),translate=[1,1,1], ...
>  rotate=xrotate(10°)+yrotate(10°), scale=1.2);
\end{eulerprompt}
\begin{eulercomment}
Lalu kita ambil selisih kedua benda tersebut.
\end{eulercomment}
\begin{eulerprompt}
>writeln(povdifference(c1,c2));
\end{eulerprompt}
\begin{eulercomment}
Sekarang tambahkan tiga sumbu.
\end{eulercomment}
\begin{eulerprompt}
>writeAxis(-1.2,1.2,axis=1); ...
>writeAxis(-1.2,1.2,axis=2); ...
>writeAxis(-1.2,1.2,axis=4); ...
>povend();
\end{eulerprompt}
\eulerimg{27}{images/EMT4Plot3D_Ghifa Attaya Ulhaq_22305144038-078.png}
\eulerheading{Fungsi Implisit}
\begin{eulercomment}
Povray dapat memplot himpunan di mana f(x,y,z)=0, seperti parameter
implisit di plot3d. Namun hasilnya terlihat jauh lebih baik.

Sintaks untuk fungsinya sedikit berbeda. Anda tidak dapat menggunakan
keluaran ekspresi Maxima atau Euler.

\end{eulercomment}
\begin{eulerformula}
\[
((x^2+y^2-c^2)^2+(z^2-1)^2)*((y^2+z^2-c^2)^2+(x^2-1)^2)*((z^2+x^2-c^2)^2+(y^2-1)^2)=d
\]
\end{eulerformula}
\begin{eulerprompt}
>povstart(angle=70°,height=50°,zoom=4);
>c=0.1; d=0.1; ...
>writeln(povsurface("(pow(pow(x,2)+pow(y,2)-pow(c,2),2)+pow(pow(z,2)-1,2))*(pow(pow(y,2)+pow(z,2)-pow(c,2),2)+pow(pow(x,2)-1,2))*(pow(pow(z,2)+pow(x,2)-pow(c,2),2)+pow(pow(y,2)-1,2))-d",povlook(red))); ...
>povend();
\end{eulerprompt}
\begin{euleroutput}
  object \{
  isosurface \{
  function \{ (pow(pow(x,2)+pow(y,2)-pow(c,2),2)+pow(pow(z,2)-1,2))*(pow(pow(y,2)+pow(z,2)-pow(c,2),2)+pow(pow(x,2)-1,2))*(pow(pow(z,2)+pow(x,2)-pow(c,2),2)+pow(pow(y,2)-1,2))-d \}
  max_gradient 5
  open
  contained_by \{ box \{ <-1,-1,-1>, <1,1,1>
   \} \}
   texture \{ pigment \{ color rgb <0.564706,0.0627451,0.0627451> \}  \} 
   finish \{ ambient 0.2 \} 
  \}\}
\end{euleroutput}
\eulerimg{27}{images/EMT4Plot3D_Ghifa Attaya Ulhaq_22305144038-080.png}
\begin{eulerprompt}
>povstart(angle=25°,height=10°); 
>writeln(povsurface("pow(x,2)+pow(y,2)*pow(z,2)-1",povlook(blue),povbox(-2,2,"")));
>povend();
\end{eulerprompt}
\eulerimg{27}{images/EMT4Plot3D_Ghifa Attaya Ulhaq_22305144038-081.png}
\begin{eulerprompt}
>povstart(angle=70°,height=50°,zoom=4);
\end{eulerprompt}
\begin{eulercomment}
Buat permukaan implisit. Perhatikan sintaksis yang berbeda dalam
ekspresi.
\end{eulercomment}
\begin{eulerprompt}
>writeln(povsurface("pow(x,2)*y-pow(y,3)-pow(z,2)",povlook(green))); ...
>writeAxes(); ...
>povend();
\end{eulerprompt}
\eulerimg{27}{images/EMT4Plot3D_Ghifa Attaya Ulhaq_22305144038-082.png}
\eulerheading{Objek Mesh}
\begin{eulercomment}
Dalam contoh ini, kita akan menunjukkan bagaimana cara membuat objek
mesh, dan menggambarkannya dengan informasi tambahan.

Kita ingin memaksimalkan xy dengan syarat x+y=1 dan mendemonstrasikan
sentuhan tangensial dari garis level.
\end{eulercomment}
\begin{eulerprompt}
>povstart(angle=-10°,center=[0.5,0.5,0.5],zoom=7);
\end{eulerprompt}
\begin{eulercomment}
Kita tidak dapat menyimpan objek dalam bentuk string seperti
sebelumnya, karena terlalu besar. Jadi, kita mendefinisikan objek
dalam file Povray menggunakan #declare. Fungsi povtriangle()
melakukannya secara otomatis. Ini dapat menerima vektor normal seperti
pov3d().

Berikut ini mendefinisikan objek jaringan (mesh), dan langsung
menulisnya ke dalam file.
\end{eulercomment}
\begin{eulerprompt}
>x=0:0.02:1; y=x'; z=x*y; vx=-y; vy=-x; vz=1;
>mesh=povtriangles(x,y,z,"",vx,vy,vz);
\end{eulerprompt}
\begin{eulercomment}
Sekarang kita mendefinisikan dua cakram, yang akan berpotongan dengan
permukaan.
\end{eulercomment}
\begin{eulerprompt}
>cl=povdisc([0.5,0.5,0],[1,1,0],2); ...
>ll=povdisc([0,0,1/4],[0,0,1],2);
\end{eulerprompt}
\begin{eulercomment}
Tulis permukaannya dikurangi kedua cakram.
\end{eulercomment}
\begin{eulerprompt}
>writeln(povdifference(mesh,povunion([cl,ll]),povlook(green)));
\end{eulerprompt}
\begin{eulercomment}
Tuliskan kedua perpotongan tersebut.
\end{eulercomment}
\begin{eulerprompt}
>writeln(povintersection([mesh,cl],povlook(red))); ...
>writeln(povintersection([mesh,ll],povlook(gray)));
\end{eulerprompt}
\begin{eulercomment}
Tulis poin maksimal.
\end{eulercomment}
\begin{eulerprompt}
>writeln(povpoint([1/2,1/2,1/4],povlook(gray),size=2*defaultpointsize));
\end{eulerprompt}
\begin{eulercomment}
Tambahkan sumbu dan selesai.
\end{eulercomment}
\begin{eulerprompt}
>writeAxes(0,1,0,1,0,1,d=0.015); ...
>povend();
\end{eulerprompt}
\eulerimg{27}{images/EMT4Plot3D_Ghifa Attaya Ulhaq_22305144038-083.png}
\eulerheading{Anaglif dalam Povray}
\begin{eulercomment}
Untuk menghasilkan anaglif untuk kacamata merah/cyan, Povray harus
dijalankan dua kali dari posisi kamera yang berbeda. Ini menghasilkan
dua file Povray dan dua file PNG, yang dimuat dengan fungsi
loadanaglyph().

Tentu saja, Anda memerlukan kacamata merah/cyan untuk melihat
contoh-contoh berikut dengan benar.

Fungsi pov3d() memiliki sakelar sederhana untuk menghasilkan anaglif.
\end{eulercomment}
\begin{eulerprompt}
>pov3d("-exp(-x^2-y^2)/2",r=2,height=45°,>anaglyph, ...
>  center=[0,0,0.5],zoom=3.5);
\end{eulerprompt}
\eulerimg{27}{images/EMT4Plot3D_Ghifa Attaya Ulhaq_22305144038-084.png}
\begin{eulercomment}
Jika Anda membuat adegan dengan objek, Anda perlu memasukkan pembuatan
adegan ke dalam fungsi, dan menjalankannya dua kali dengan nilai
berbeda untuk parameter anaglyph.
\end{eulercomment}
\begin{eulerprompt}
>function myscene ...
\end{eulerprompt}
\begin{eulerudf}
    s=povsphere(povc,1);
    cl=povcylinder(-povz,povz,0.5);
    clx=povobject(cl,rotate=xrotate(90°));
    cly=povobject(cl,rotate=yrotate(90°));
    c=povbox([-1,-1,0],1);
    un=povunion([cl,clx,cly,c]);
    obj=povdifference(s,un,povlook(red));
    writeln(obj);
    writeAxes();
  endfunction
\end{eulerudf}
\begin{eulercomment}
Fungsi povanaglyph() melakukan semua ini. Parameternya seperti
gabungan povstart() dan povend().
\end{eulercomment}
\begin{eulerprompt}
>povanaglyph("myscene",zoom=4.5);
\end{eulerprompt}
\eulerimg{27}{images/EMT4Plot3D_Ghifa Attaya Ulhaq_22305144038-085.png}
\eulerheading{Mendefinisikan Objek sendiri}
\begin{eulercomment}
Antarmuka povray Euler berisi banyak objek. Namun Anda tidak dibatasi
pada hal ini. Anda dapat membuat objek sendiri, yang menggabungkan
objek lain, atau merupakan objek yang benar-benar baru.

Kita mendemonstrasikan torus. Perintah Povray untuk ini adalah
"torus". Jadi kita mengembalikan string dengan perintah ini dan
parameternya. Perhatikan bahwa torus selalu berpusat pada titik asal.
\end{eulercomment}
\begin{eulerprompt}
>function povdonat (r1,r2,look="") ...
\end{eulerprompt}
\begin{eulerudf}
    return "torus \{"+r1+","+r2+look+"\}";
  endfunction
\end{eulerudf}
\begin{eulercomment}
Ini torus pertama kita.
\end{eulercomment}
\begin{eulerprompt}
>t1=povdonat(0.8,0.2)
\end{eulerprompt}
\begin{euleroutput}
  torus \{0.8,0.2\}
\end{euleroutput}
\begin{eulercomment}
Mari kita gunakan objek ini untuk membuat torus kedua, diterjemahkan
dan diputar.
\end{eulercomment}
\begin{eulerprompt}
>t2=povobject(t1,rotate=xrotate(90°),translate=[0.8,0,0])
\end{eulerprompt}
\begin{euleroutput}
  object \{ torus \{0.8,0.2\}
   rotate 90 *x 
   translate <0.8,0,0>
   \}
\end{euleroutput}
\begin{eulercomment}
Sekarang kita tempatkan objek-objek tersebut ke dalam sebuah adegan.
Untuk tampilannya kami menggunakan Phong Shading.
\end{eulercomment}
\begin{eulerprompt}
>povstart(center=[0.4,0,0],angle=0°,zoom=3.8,aspect=1.5); ...
>writeln(povobject(t1,povlook(green,phong=1))); ...
>writeln(povobject(t2,povlook(green,phong=1))); ...
\end{eulerprompt}
\begin{eulerttcomment}
 >povend();
\end{eulerttcomment}
\begin{eulercomment}
memanggil program Povray. Namun, jika terjadi kesalahan, program ini
tidak akan menampilkan pesan kesalahan. Oleh karena itu, Anda
sebaiknya menggunakan

\end{eulercomment}
\begin{eulerttcomment}
 >povend(<exit);
\end{eulerttcomment}
\begin{eulercomment}

jika ada yang tidak berfungsi. Ini akan membuat jendela Povray tetap
terbuka.
\end{eulercomment}
\begin{eulerprompt}
>povend(h=320,w=480);
\end{eulerprompt}
\eulerimg{18}{images/EMT4Plot3D_Ghifa Attaya Ulhaq_22305144038-086.png}
\begin{eulercomment}
Berikut adalah contoh yang lebih rinci. Kita menyelesaikan

\end{eulercomment}
\begin{eulerformula}
\[
Ax \le b, \quad x \ge 0, \quad c.x \to \text{Max.}
\]
\end{eulerformula}
\begin{eulercomment}
dan menunjukkan titik-titik yang memungkinkan dan optimum dalam plot
3D.
\end{eulercomment}
\begin{eulerprompt}
>A=[10,8,4;5,6,8;6,3,2;9,5,6];
>b=[10,10,10,10]';
>c=[1,1,1];
\end{eulerprompt}
\begin{eulercomment}
Pertama, mari kita periksa, apakah contoh ini punya solusinya.
\end{eulercomment}
\begin{eulerprompt}
>x=simplex(A,b,c,>max,>check)'
\end{eulerprompt}
\begin{euleroutput}
  [0,  1,  0.5]
\end{euleroutput}
\begin{eulercomment}
Ya, itu sudah ada.

Selanjutnya, kita mendefinisikan dua objek. Yang pertama adalah bidang

\end{eulercomment}
\begin{eulerformula}
\[
a \cdot x \le b
\]
\end{eulerformula}
\begin{eulerprompt}
>function oneplane (a,b,look="") ...
\end{eulerprompt}
\begin{eulerudf}
    return povplane(a,b,look)
  endfunction
\end{eulerudf}
\begin{eulercomment}
Kemudian kita mendefinisikan perpotongan semua setengah ruang dan
sebuah kubus.
\end{eulercomment}
\begin{eulerprompt}
>function adm (A, b, r, look="") ...
\end{eulerprompt}
\begin{eulerudf}
    ol=[];
    loop 1 to rows(A); ol=ol|oneplane(A[#],b[#]); end;
    ol=ol|povbox([0,0,0],[r,r,r]);
    return povintersection(ol,look);
  endfunction
\end{eulerudf}
\begin{eulercomment}
Sekarang kita dapat merencanakan adegannya.
\end{eulercomment}
\begin{eulerprompt}
>povstart(angle=120°,center=[0.5,0.5,0.5],zoom=3.5); ...
>writeln(adm(A,b,2,povlook(green,0.4))); ...
>writeAxes(0,1.3,0,1.6,0,1.5); ...
\end{eulerprompt}
\begin{eulercomment}
Berikut ini adalah lingkaran di sekitar optimal.
\end{eulercomment}
\begin{eulerprompt}
>writeln(povintersection([povsphere(x,0.5),povplane(c,c.x')], ...
>  povlook(red,0.9)));
\end{eulerprompt}
\begin{eulercomment}
Dan kesalahan ke arah optimal.
\end{eulercomment}
\begin{eulerprompt}
>writeln(povarrow(x,c*0.5,povlook(red)));
\end{eulerprompt}
\begin{eulercomment}
Kita menambahkan teks ke layar. Teks hanyalah objek 3D. Kita perlu
menempatkan dan memutarnya sesuai dengan pandangan kita.
\end{eulercomment}
\begin{eulerprompt}
>writeln(povtext("Linear Problem",[0,0.2,1.3],size=0.05,rotate=5°)); ...
>povend();
\end{eulerprompt}
\eulerimg{27}{images/EMT4Plot3D_Ghifa Attaya Ulhaq_22305144038-089.png}
\eulerheading{Contoh lain}
\begin{eulercomment}
Anda dapat menemukan beberapa contoh lebih lanjut untuk Povray di
Euler dalam file-file berikut.

See: Examples/Dandelin Spheres\\
See: Examples/Donat Math\\
See: Examples/Trefoil Knot\\
See: Examples/Optimization by Affine Scaling

\begin{eulercomment}
\eulerheading{Latihan Soal}
\begin{eulercomment}
1.Gambarlah grafik dari fungsi berikut.\\
\end{eulercomment}
\begin{eulerformula}
\[
f(x,y)=x^2+3y^2
\]
\end{eulerformula}
\begin{eulercomment}
Jawab :
\end{eulercomment}
\begin{eulerprompt}
>plot3d("x^2+3*y^2",n=40,grid=2):
\end{eulerprompt}
\eulerimg{27}{images/EMT4Plot3D_Ghifa Attaya Ulhaq_22305144038-091.png}
\begin{eulercomment}
2. Gambarlah grafik dari fungsi berikut\\
\end{eulercomment}
\begin{eulerformula}
\[
f(x,y)= (2x^2+y^2)e^{x^2-y^2}
\]
\end{eulerformula}
\begin{eulercomment}
Jawab :
\end{eulercomment}
\begin{eulerprompt}
>plot3d("(2*x^2+y^2)*E^(x^2-y^2)",scale=\{1,2\},xmin=-5,xmax=5,ymin=-7,ymax=7,frame=4):
\end{eulerprompt}
\eulerimg{27}{images/EMT4Plot3D_Ghifa Attaya Ulhaq_22305144038-093.png}
\begin{eulercomment}
3. Gambarkan grafik fungsi logaritma berikut\\
\end{eulercomment}
\begin{eulerformula}
\[
f(x,y)=log(x^2+2y^2)
\]
\end{eulerformula}
\begin{eulercomment}
Jawab :
\end{eulercomment}
\begin{eulerprompt}
>plot3d("log(x^2+2*y^2)"):
\end{eulerprompt}
\eulerimg{27}{images/EMT4Plot3D_Ghifa Attaya Ulhaq_22305144038-095.png}
\begin{eulercomment}
4. Gambarkan grafik fungsi dari\\
\end{eulercomment}
\begin{eulerformula}
\[
g(x,y)=sin(2x).cos(2y)
\]
\end{eulerformula}
\begin{eulercomment}
Jawab :
\end{eulercomment}
\begin{eulerprompt}
>plot3d("sin(2*x)*cos(2*y)"):
\end{eulerprompt}
\eulerimg{27}{images/EMT4Plot3D_Ghifa Attaya Ulhaq_22305144038-097.png}
\begin{eulercomment}
5. Bagaimana bentuk atau pola dari objek persamaan berikut

\end{eulercomment}
\begin{eulerformula}
\[
f(x,y) = cos(x)sin(y)
\]
\end{eulerformula}
\begin{eulerformula}
\[
g(x,y) = sin(x)cos(y)
\]
\end{eulerformula}
\begin{eulerformula}
\[
h(x,y) = cos(x)
\]
\end{eulerformula}
\begin{eulercomment}
dengan\\
\end{eulercomment}
\begin{eulerformula}
\[
0 \leq x \leq 2\pi
\]
\end{eulerformula}
\begin{eulercomment}
\end{eulercomment}
\begin{eulerformula}
\[
-\frac{\pi}{2} \leq y \leq \frac{\pi}{2}
\]
\end{eulerformula}
\begin{eulercomment}
Jawab :
\end{eulercomment}
\begin{eulerprompt}
>plot3d("cos(x)*sin(y)","sin(x)*cos(y)","cos(x)", a=0, b=2*pi, c=pi/2, d=-pi/2, ...
>>hue,color=red, light=[0,1,0],<frame, ...
>n=90, grid=[15,30],wirecolor=green,zoom=3):
\end{eulerprompt}
\eulerimg{27}{images/EMT4Plot3D_Ghifa Attaya Ulhaq_22305144038-103.png}
\begin{eulercomment}
6. Gambarkan fungsi dari\\
\end{eulercomment}
\begin{eulerformula}
\[
A=\{(x,y,z):x^3+2y^3+3z^3=1\}
\]
\end{eulerformula}
\begin{eulercomment}
Jawab :
\end{eulercomment}
\begin{eulerprompt}
>plot3d("x^3+2*y^3+3*z^3-1",r=5,implicit=3):
\end{eulerprompt}
\eulerimg{27}{images/EMT4Plot3D_Ghifa Attaya Ulhaq_22305144038-105.png}
\end{eulernotebook}
\end{document}
